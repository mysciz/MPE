\titleformat{\chapter}[hang]{\linespread{1}\heiti\sanhao\bfseries\filright}{\thechapter}{1em}{}{}
\chapter{LLT模型}
\section{模型概览}
2003年,Marius Lysaker等人提出了一种能够有效抑制ROF模型阶梯效应的LLT模型\cite{lysaker2003noise}:
\begin{equation}
    \min _{u}\{J(u)=\alpha \int _{\Omega}|D^{2}u|dxdy+\frac{1}{2}||u-f||^{2}\}.
    \tag{16}
\end{equation}
其中,$u$为原始图像(未知);$|D^{2}u|=\sqrt{u_{xx}^{2}+u_{xy}^{2}+u_{yx}^{2}+u_{yy}^{2}}$;$f$为已知的观察图像;$\Omega$是有界凸集;$\alpha$是用来调节光滑项和拟合项的正则化参数,$\alpha$较大时恢复图像比较光滑,$\alpha$较小时拟合效果好。

在LLT模型中,二阶梯度被用作正则化项,以平滑图像并减少噪声的影响,同时保留边缘信息。但由于涉及二阶梯度,其计算通常比一阶梯度更为复杂,需要更多的计算资源和时间。所以,在LLT模型求解过程中我们使用不动点迭代法。下面给出不动点迭代法的算法步骤\cite{rezaiee2023evaluation}:

可知LLT模型的Euler-Lagrange方程为
\begin{equation}
    g(u)=\alpha (\frac{u_{xx}}{|D^{2}u|})+\frac{(u_{yx})}{|D^{2}u|})_{yx}+(\frac{u_{xy}}{|D^{2}u|})_{xy}+\frac{(u_{yy}}{|D^{2}u|})_{yy}]+(u-f)=0.
    \tag{17}
\end{equation}
选择一个初始猜测 $u^0$,通常可以选择观察到的图像 $f$ 作为初始值。对于每一步迭代 $k$,使用下面的迭代公式更新 $u$:
\begin{equation}
    u^{k+1} = \text{prox}_{\lambda \alpha |D^2|}(u^k + \lambda (u^k - f))
    \tag{18}
\end{equation}
其中 $\text{prox}$ 是近端算子,定义为:
\begin{equation}
    \text{prox}_{\lambda \alpha |D^2|}(v) = \arg\min_u \left\{ \frac{1}{2\lambda} \|u - v\|^2 + \alpha \int_{\Omega} |D^2 u| \, dxdy \right\}.
    \tag{19}
\end{equation}
这个近端算子的计算涉及到求解一个带有二阶梯度的优化问题,通常需要数值方法来求解。
检查迭代是否收敛。如果 $u^{k+1}$ 与 $u^k$ 之间的变化非常小,或者达到了预设的最大迭代次数,则停止迭代。

显然, $\alpha$ 和 $\lambda$ 的选择对算法的性能有显著影响,在实际应用中,由于涉及到高阶导数的计算,可能会遇到数值不稳定的问题。可以通过引入适当的数值稳定技术来解决。

LLT模型在图像处理中具有广泛的应用前景,特别是在需要快速响应图像变化的应用中。然而,LLT模型可能会受到边界效应的影响,因此在实际应用中可能需要结合其他方法来进行边界处理。

\section{对比分析}
ROF模型通过最小化图像的总变分来实现去噪。这种方法在去除噪声的同时,能够较好地保护图像的边缘和纹理细节。然而,ROF模型可能会产生“锯齿效应”,即在图像边缘处出现不平滑的现象。

LLT模型通过引入二阶滤波器的概念来平滑图像数据,从而减少噪声的影响并保留边缘信息。然而,与ROF去噪模型相比,LLT模型不是通过递归的方式计算每个像素点的LLT值,而是通过求解一个涉及二阶梯度的变分问题来找到最优解。这种方法有效地减少了传统边缘检测方法的延迟问题,同时保留了边缘检测的能力。相比之下,ROF模型主要关注一阶梯度,可能在边缘保持方面略有不同,但同样需要考虑边界条件和数值稳定性问题。
