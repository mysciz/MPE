\titleformat{\chapter}[hang]{\linespread{1}\heiti\sanhao\bfseries\filright}{\thechapter}{1em}{}{}
\chapter{引言}
计算机视觉是计算机科学的“眼睛”,它赋予机器理解和解析视觉世界的能力,让计算机能够像人类一样“看”并理解图像和视频中的信息。这一领域的研究不仅涉及图像处理、机器学习、统计学等多学科知识,还广泛应用于自动驾驶、医疗影像分析、安防监控等多个领域,极大地推动了智能化进程。随着技术的不断进步,计算机视觉正逐步实现从感知到认知的跨越,为人工智能的发展提供强有力的支撑。

在这一进程中,图像处理作为计算机视觉的“灵魂”,扮演着至关重要的角色。它不仅涉及对图像质量的改善(如去噪、对比度增强等),还包括特征提取、模式识别等多个方面。通过一系列算法和技术手段,图像处理将原始图像转换成更适合进一步分析或理解的形式,为后续的目标检测、物体跟踪乃至场景理解提供了坚实的基础。在实际应用中,无论是人脸识别门禁系统、智能监控摄像头还是基于图片内容的搜索引擎,都离不开高效精准的图像处理技术的支持。

偏微分方程(PDEs)作为一种强大的数学工具,能够描述图像数据的连续变化和局部特性,通过其演化过程实现对图像的精细操控和优化。在图像处理过程中,不少模型从偏微分方程切入,同时,基于偏微分方程的图像处理模型能够高效地解决去噪、增强、分割等任务。随着深度学习等先进方法的兴起,偏微分方程与神经网络的结合也成为了研究的热点。通过将偏微分方程的连续模型与神经网络的离散计算相结合,可以开发出更加高效、精准的图像处理算法。这种跨领域的融合不仅推动了偏微分方程在图像处理中的应用发展,也为整个计算机视觉领域带来了新的创新机遇。
