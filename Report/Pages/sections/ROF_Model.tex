\titleformat{\chapter}[hang]{\linespread{1}\heiti\sanhao\bfseries\filright}{\thechapter}{1em}{}{}
\chapter{ROF模型}
\section{模型导出}
实际上,图像去噪的根本目标是将观察到的噪声图像 $ f $ 恢复成原始的清晰图像 $ u $。这个过程可以通过求解一个优化问题来实现。ROF模型的核心思想就是最小化图像的总变差,同时保持与观测数据的一致性。

在ROF模型中,定义图像 $ u $ 的梯度的 $ L^1 $ 范数为总变差(TV)范数,作为衡量图像变化程度的一个指标。\cite{rudin1994total}
\begin{equation}
    \text{TV}(u) = \int_{\Omega} |\bigtriangledown u| \, dx 
    \tag{7}
\end{equation}
其中,$ \bigtriangledown u $ 表示图像 $ u $ 的梯度,$ |\bigtriangledown u| $ 是梯度的模长,$ \Omega $ 是图像的定义域。

同时,为了确保恢复的图像与观测数据一致,需要引入数据保真项(也称为保真项或拟合项),通常使用 $ L^2 $ 范数来度量观测图像 $ f $ 和恢复图像 $ u $ 之间的差异:
\begin{equation}
||L||_2=\frac{1}{2} \int_{\Omega} (u - f)^2 \, dx 
\tag{8}
\end{equation}

结合上述两个部分,ROF模型可以表示为以下优化问题:
\begin{equation}
     \min_{u} \text{TV}(u) + \lambda \cdot \frac{1}{2} \int_{\Omega} (u - f)^2 \, dx 
     \tag{9}
\end{equation}

其中,$ \lambda $ 是一个正则化参数,用于平衡总变差最小化和数据保真性之间的权重。上式即为ROF模型的lost Funtion,也即能量泛函,通过梯度下降法、牛顿-拉夫逊法等迭代法,最小化一个能量泛函来恢复或去噪图像。

\section{数值求解}
对于ROF模型的求解,通常采用迭代法,可以使用梯度下降法(公式(10),也可以使用牛顿-拉夫逊法(公式(11),但这里给出Primal-Dual算法\cite{chen2013primal}。
\begin{equation}
     x_{n+1} = x_n - \alpha_n \bigtriangledown f(x_n) \tag{10} 
\end{equation}
\begin{equation}
        x_{n+1} = x_n - [H(x_n)]^{-1} \bigtriangledown f(x_n) \tag{11}
\end{equation}

对于Primal-Dual算法,给出以下算法步骤。

引入对偶变量 \( p \),定义拉格朗日函数:
\[ L(u, p) = \int_{\Omega} \phi(|\bigtriangledown u|) \, dx + \frac{\lambda}{2} \int_{\Omega} |u - v|^2 \, dx + \int_{\Omega} p \cdot (-\bigtriangledown u) \, dx \tag{12}\]
通过对$u$ 和 $p$ 分别求导,我们可以得到以下更新规则:

更新原始变量 \( u \):
\[ u^{k+1} = (\lambda I + \bigtriangledown \cdot)^{-1} (\lambda v + \bigtriangledown \cdot p^k)\tag{13} \] 
更新对偶变量 \( p \):
\[ p^{k+1} = p^k + \tau (\bigtriangledown u^{k+1})\tag{14} \]
其中$\tau$是对偶变量的步长。

投影步骤(如果需要):
\[ p^{k+1} = \frac{p^{k+1}}{\max(1, |p^{k+1}|)} \tag{15}\]
检查收敛性:
如果 \( u^{k+1} \) 和 \( p^{k+1} \) 的变化小于预设阈值,则认为算法已经收敛。
